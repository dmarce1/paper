\documentclass{aastex63}
\usepackage{amsmath}
\usepackage{bm}      % for \bm macro
\usepackage{natbib}
\setcitestyle{numbers}




\begin{document}


\newcommand{\ddt}[1]{\frac{\partial}{\partial t} #1} 
\newcommand{\divergence}[1]{{\bf \nabla} \cdot #1}
\newcommand{\curl}[1]{{\bf \nabla} \times #1}
\newcommand{\gradient}[1]{{\bf \nabla} #1}
\newcommand{\vect}[1]{{\bf #1}}
\newcommand{\am}{\bm{l}}


\section{Model Equations}
The evolution equations are
\begin{equation}
\ddt{\rho}+\divergence{\rho\vect{v}}=0,
\label{rho}
\end{equation}
\begin{equation}
\ddt{\vect{s}}+\divergence{\vect{v}\vect{s} + \gradient{p}}=\rho \vect{g},
\label{rhou}
\end{equation}
\begin{equation}
\ddt{\am}+\divergence{\vect{v}\am - \curl{\vect{x} p}}=\vect{x}\times\rho \vect{g},
\label{am}
\end{equation}
\begin{equation}
\ddt{\left(E+\tfrac{1}{2}\rho\phi\right)}+\divergence{\vect{v}\left(E+p+\rho\phi\right)}=\tfrac{1}{2}\left(\phi\ddt{\rho}-\rho\ddt{\phi}\right),
\label{E}
\end{equation}
and
\begin{equation}
\ddt{\tau}+\divergence{\tau\vect{u}}=0,
\label{tau}
\end{equation}
where $t$ is the time coordinate, $\rho$ is the mass density, $\vect{u}$ is the velocity, $\vect{s}$ is the inertial frame momentum density, $p$ is the gas pressure, 
$\vect{g}$ is the gravitational acceleration, 
$\am$ is the angular momentum relative to the coordinate center, 
$\vect{x}$ is the spatial coordinate relative to the origin, 
$E$ is the gas energy, $\phi$ is the gravitational potential, 
and $\tau$ is the entropy tracer.

The gas pressure, $p$, is taken as the ideal gas pressure for a monatomic gas, 
\begin{equation}
p = \left( \gamma - 1 ) \right) e, 
\end{equation}
where $\gamma = \tfrac{5}{3}$. The internal gas energy density, $e$, is computed using the dual energy formalism of \cite{BRYAN1995},
\begin{equation}
    e = 
\begin{cases}
    E - \tfrac{1}{2} \rho u^2, & \text{if } E - \tfrac{1}{2} \rho u^2 \geq \epsilon_1 E\\
    \tau^{\gamma},             & \text{else}
\end{cases},
\end{equation}
where $\epsilon_1 = 0.001$.

The angular momentum is defined as
\begin{equation}
\am = \vect{x} \times \rho \vect{u}.
\end{equation}


\section{Method}
\cite{DESPRES2015} treated the angular momentum fields as part of the sub-cell velocity structure. This couples the angular momentum to the linear momentum, allowing any excess or 
deficit of angular momentum in a given cell to feed back into the system's linear momentum in a way that preserves the total angular momentum. They used a 2nd order MUSCL reconstruction.
Here we detail our modification to the reconstruction of the piece-wise parabolic method (PPM) \cite{COLELLA1984}, extending the method of \cite{DESPRES2015} to a fully three-dimensional 
and 3rd order reconstruction.
 
We begin by reconstructing cell averaged values on the surface of the cells. Rather than reconstruct the vector of conserved quanties, 
 \begin{align}
 U &= \begin{bmatrix}
 	\rho \\
        \vect{s} \\
        \am \\
        E \\
        \tau
      \end{bmatrix},
\end{align}
we reconstruct 
 \begin{align}
 V &= \begin{bmatrix}
 	\rho \\
        \vect{u}\\
        \vect{z} \\
        E - \tfrac{1}{2} \rho u^2\\
        \tau
      \end{bmatrix},
\end{align}
where $\vect{z} := \frac{\am}{\rho} - \vect{x_c} \times \vect{s}$ is the spin angular momentum relative to a particular grid cell with center $\vect{x_c}$,
and then transform back to $U$ after reconstruction of $V$.

We use PPM to reconstruct values at each face, edge, and vertex. Here, we denote particlar faces, edges, and vertices using a $+$, $0$, or $-$ superscript
for each dimension. For example, $a^{+00}_{j k l} := a_{j k l}\left(x_{j+\tfrac{1}{2} k l}\right)$ is the value of $a$ at the right x-face of the $ijk^\mathrm{th}$ cell. 
The five cell stencil required for PPM is
taken along the direction of each face, edge, or vertex. For example, computing $a^{+0-}_{j k l}$ uses the cell averaged values $a_{j + 2 k l - 2}$,$a_{j + 1 k l - 1}$,
$a_{j k l}$,$u_{j - 1 k l + 1}$, and
$a_{j - 2 k l + 2}$.  We implement the PPM reconstruction as described in \cite{COLELLA1984}, including contact discontinuity detection, with one difference: before imposing monotonicity constraints
on the velocity field, we modify the reconstructed values to match as closely as possible the evolved angular momemtum field. 

First we measure the amount of angular momentum, relative to cell centers, represented by the reconstructed linear momenta.
Summing over all $\pm$ combinations using Simpson's rule, the measured angular momentum for a cell is
\begin{equation}
\begin{split}
\vect{z}_{M, j k l} := 
& \frac{1}{216} \Sigma^{\pm\pm\pm}  \left(\vect{x}_{j\pm\tfrac{1}{2} k \pm\tfrac{1}{2} l\pm\tfrac{1}{2}}  - \vect{x}_{j k l}\right)\times \vect{\rho}^{\pm\pm\pm}_{j k l} \vect{u}^{\pm\pm\pm}_{j k l} + \\
& \frac{1}{54}  \ \Sigma^{\pm\pm} \left(\vect{x}_{j\pm\tfrac{1}{2}  k \pm\tfrac{1}{2} l}                  - \vect{x}_{j k l}\right)\times \vect{\rho}^{\pm\pm 0}_{j k l}  \vect{u}^{\pm\pm 0}_{j k l}  + \\ 
& \frac{1}{54}  \ \Sigma^{\pm\pm} \left(\vect{x}_{j\pm\tfrac{1}{2}  k                 l\pm\tfrac{1}{2}}   - \vect{x}_{j k l}\right)\times \vect{\rho}^{\pm 0\pm}_{j k l}  \vect{u}^{\pm 0\pm}_{j k l}  + \\
& \frac{1}{54}  \ \Sigma^{\pm\pm} \left(\vect{x}_{j                 k\pm\tfrac{1}{2}  l\pm\tfrac{1}{2}}   - \vect{x}_{j k l}\right)\times \vect{\rho}^{0\pm\pm }_{j k l}  \vect{u}^{0\pm\pm }_{j k l}  + \\
& \frac{2}{27}  \ \Sigma^{\pm}   \left(\vect{x}_{j\pm\tfrac{1}{2} k l}                                    - \vect{x}_{j k l}\right)\times \vect{\rho}^{\pm 0 0}_{j k l}   \vect{u}^{\pm 0 0}_{j k l}   +  \\
& \frac{2}{27}  \ \Sigma^{\pm}   \left(\vect{x}_{j k\pm\tfrac{1}{2} l}                                    - \vect{x}_{j k l}\right)\times \vect{\rho}^{0 \pm 0}_{j k l}   \vect{u}^{0 \pm 0}_{j k l}   +  \\
& \frac{2}{27}  \ \Sigma^{\pm}   \left(\vect{x}_{j k l\pm\tfrac{1}{2}}                                    - \vect{x}_{j k l}\right)\times \vect{\rho}^{0 0 \pm}_{j k l}   \vect{u}^{0 0 \pm}_{j k l}   
\end{split}
\end{equation}

Taking a Taylor expansion of the angular momentum relative to a cell center and integrating over the whole cell, we find
\begin{equation}
{\vect{z}}_{j k l} = \tfrac{1}{12} \epsilon_{n m q} \left(\frac{\partial s_{m}}{\partial x_n}\right)_{j k l} \Delta^2 \vect{e}_q + \vect{\mathcal{O}}(\Delta^3),
\end{equation}
where $\Delta$ is the even grid spacing and $\vect{e}_q$ is the $q^\mathrm{th}$ unit vector. Using this fact, we can correct the linear momenta to reproduce the
correct angular momenta by modifying the reconstruction according to
\begin{equation}
\vect{u}^{* \pm\pm\pm}_{j k l} = \vect{u}^{\pm\pm\pm}_{j k l} + \frac{12}{\Delta^2} \frac{\vect{z}_{j k l} - \vect{z}_{M, j k l}}{\rho^{\pm\pm\pm}_{j k l}+\rho^{\mp\mp\mp}_{j k l}} \times \left( \vect{x}_{j\pm\tfrac{1}{2}k\pm\tfrac{1}{2}l\pm\tfrac{1}{2}} - \vect{x}_{j k l} \right),
\label{beg_cor}
\end{equation}
\begin{equation}
\vect{u}^{* 0\pm\pm}_{j k l} = \vect{u}^{0\pm\pm}_{j k l} + \frac{12}{\Delta^2} \frac{\vect{z}_{j k l} - \vect{z}_{M, j k l}}{\rho^{0\pm\pm}_{j k l}+\rho^{0\mp\mp}_{j k l}} \times \left( \vect{x}_{j k\pm\tfrac{1}{2}l\pm\tfrac{1}{2}} - \vect{x}_{j k l} \right),
\end{equation}
\begin{equation}
\vect{u}^{* \pm 0\pm}_{j k l} = \vect{u}^{\pm 0\pm}_{j k l} + \frac{12}{\Delta^2} \frac{\vect{z}_{j k l} - \vect{z}_{M, j k l}}{\rho^{\pm 0\pm}_{j k l}+\rho^{\mp 0\mp}_{j k l}} \times \left( \vect{x}_{j\pm\tfrac{1}{2} k l\pm\tfrac{1}{2}} - \vect{x}_{j k l} \right),
\end{equation}
\begin{equation}
\vect{u}^{* \pm\pm 0}_{j k l} = \vect{u}^{\pm\pm 0}_{j k l} + \frac{12}{\Delta^2} \frac{\vect{z}_{j k l} - \vect{z}_{M, j k l}}{\rho^{\pm\pm 0}_{j k l}+\rho^{\mp\mp 0}_{j k l}} \times \left( \vect{x}_{j\pm\tfrac{1}{2} k\pm\tfrac{1}{2}l} - \vect{x}_{j k l} \right),
\end{equation}
\begin{equation}
\vect{u}^{* 0 0\pm}_{j k l} = \vect{u}^{0 0\pm}_{j k l} + \frac{12}{\Delta^2} \frac{\vect{z}_{j k l} - \vect{z}_{M, j k l}}{\rho^{0 0\pm}_{j k l}+\rho^{0 0\mp}_{j k l}} \times \left( \vect{x}_{j k l\pm\tfrac{1}{2}} - \vect{x}_{j k l} \right),
\end{equation}
\begin{equation}
\vect{u}^{* 0 \pm 0}_{j k l} = \vect{u}^{0\pm 0}_{j k l} + \frac{12}{\Delta^2} \frac{\vect{z}_{j k l} - \vect{z}_{M, j k l}}{\rho^{0\pm 0}_{j k l}+\rho^{0 \mp 0}_{j k l}} \times \left( \vect{x}_{j k\pm\tfrac{1}{2} l} - \vect{x}_{j k l} \right),
\end{equation}
\begin{equation}
\vect{u}^{* \pm 0 0}_{j k l} = \vect{u}^{\pm 0 0}_{j k l} + \frac{12}{\Delta^2} \frac{\vect{z}_{j k l} - \vect{z}_{M, j k l}}{\rho^{\pm 0 0}_{j k l}+\rho^{\mp 0 0}_{j k l}} \times \left( \vect{x}_{j\pm\tfrac{1}{2} k l} - \vect{x}_{j k l} \right),
\label{end_cor}
\end{equation}

Equations \ref{beg_cor} through \ref{end_cor} will reproduce a momentum field that matches the evolved angular momentum, however the new reconstruction may introduce new extrema to the solution. We remove
these extrema in two steps. First we limit the slope of the new reconstruction according to
\begin{equation}
\label{step1}
a_i^{* \pm} \rightarrow a_i^{* \pm} \pm \mathrm{maxmod}\left( a_i^+ - a_i^-, \mathrm{maxmod}\left(
\mathrm{minmod}\left(a_{i+1} - a_i, 2\left(a_i - a_{i-1}\right)\right),
\mathrm{minmod}\left( 2 \left(a_{i+1} - a_i\right),a_i - a_{i-1}\right)
\right)\right),
\end{equation}
where
\begin{equation}
\mathrm{minmod}\left(p, q\right) := \tfrac{1}{2}\left(\mathrm{sgn}\left(p\right)+\mathrm{sgn}\left(q\right)\right) \mathrm{min}\left(\left|p\right|,\left|q\right|\right),
\end{equation}
and
\begin{equation}
\mathrm{maxmod}\left(p, q\right) := \tfrac{1}{2}\left(\mathrm{sgn}\left(p\right)+\mathrm{sgn}\left(q\right)\right) \mathrm{max}\left(\left|p\right|,\left|q\right|\right).
\end{equation}
Equation \ref{step1} limits the slope of the new reconstruction to the least restrictive of either the  maximum allowed for 2nd order TVD or the slope produced by the application of PPM.
The resulting reconstruction may still not be TVD stable, however, because PPM is a third order scheme. To account for this we modify $a_i^{* \pm}$ according to
\begin{equation}
a_i^{+} \rightarrow  
\begin{cases}
   a_{i+1}                      , & \left(a_i^{*+} - a_i\right)\left(a_{i+1} - a_i^{*+}\right) < 0 \\
   a^{* +} - a^{* -}_i + a_{i-1}, & \left(a_i^{*-} - a_i\right)\left(a_{i-1} - a_i^{*-}\right) < 0 \\
   a^{*+}                       , & \text{else}
\end{cases},
\end{equation}
\begin{equation}
a_i^{+} \rightarrow  
\begin{cases}
   a^{* -} - a^{* +}_i + a_{i+1}, & \left(a_i^{*+} - a_i\right)\left(a_{i+1} - a_i^{*+}\right) < 0 \\
   a_{i-1}                      , & \left(a_i^{*-} - a_i\right)\left(a_{i-1} - a_i^{*-}\right) < 0 \\
   a^{*-}                       , & \text{else}
\end{cases}.
\end{equation}
This effectively limits the parabolic component of the reconstructed scheme, pushing the reconstructed values to the range of values in the two neigboring cells as needed. 

After the above procedure, we apply the monotonicity constraints of the PPM method to the momentum field.

\bibliographystyle{plain}
\bibliography{references}


\end{document}
