\documentclass{aastex63}
\usepackage{amsmath}
\usepackage{bm}      % for \bm macro
\usepackage{natbib}
\setcitestyle{numbers}




\begin{document}


\newcommand{\ddt}[1]{\frac{\partial}{\partial t} #1} 
\newcommand{\divergence}[1]{{\bf \nabla} \cdot #1}
\newcommand{\curl}[1]{{\bf \nabla} \times #1}
\newcommand{\gradient}[1]{{\bf \nabla} #1}
\newcommand{\vect}[1]{{\bf #1}}
\newcommand{\am}{\bm{l}}

The evolution equations are
\begin{equation}
\ddt{\rho}+\divergence{\rho\vect{u}}=0,
\label{rho}
\end{equation}
\begin{equation}
\ddt{\rho \vect{u}}+\divergence{\rho\vect{u}\vect{u} + \gradient{p}}=\rho \vect{g},
\label{rhou}
\end{equation}
\begin{equation}
\ddt{\am}+\divergence{\vect{u}\am - \curl{\vect{x} p}}=\vect{x}\times\rho \vect{g},
\label{am}
\end{equation}
\begin{equation}
\ddt{\left(E+\tfrac{1}{2}\rho\phi\right)}+\divergence{\vect{u}\left(E+p+\rho\phi\right)}=\tfrac{1}{2}\left(\phi\ddt{\rho}-\rho\ddt{\phi}\right),
\label{E}
\end{equation}
and
\begin{equation}
\ddt{\tau}+\divergence{\tau\vect{u}}=0,
\label{tau}
\end{equation}
where $t$ is the time coordinate, $\rho$ is the mass density, $\vect{u}$ is the velocity, $\vect{x}$ is the spatial coordinate relative to the origin, 
$p$ is the gas pressure, $\vect{g}$ is the gravitational acceleration, 
$\am$ is the angular momentum relative to the coordinate center, $E$ is the gas energy, $\phi$ is the gravitational potential, and $\tau$ is the entropy
tracer.

The gas pressure, $p$, is taken as the ideal gas pressure for a monatomic gas, 
\begin{equation}
p = \left( \gamma - 1 ) \right) e, 
\end{equation}
where $\gamma = \tfrac{5}{3}$. The internal gas energy density, $e$, is computed using the dual energy formalism of (CITE),
\begin{equation}
    e = 
\begin{cases}
    E - \tfrac{1}{2} \rho u^2, & \text{if } E - \tfrac{1}{2} \rho u^2 \geq \epsilon_1 E\\
    \tau^{\gamma},             & \text{else}
\end{cases},
\end{equation}
where $\epsilon_1 = 0.001$.

The angular momentum, defined as
\begin{equation}
\am = \vect{x} \times \rho \vect{u}.
\end{equation}


\section{Method}
We begin by reconstructing cell averaged values on faces using the piece-wise parabolic method (PPM) of \cite{COLELLA1984}.

\bibliographystyle{plain}
\bibliography{references}


\end{document}
