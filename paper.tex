\documentclass{aastex63}
\usepackage{amsmath}
\usepackage{bm}      % for \bm macro
\usepackage{natbib}
\setcitestyle{numbers}




\begin{document}


\newcommand{\ddt}[1]{\frac{\partial}{\partial t} #1} 
\newcommand{\divergence}[1]{{\bf \nabla} \cdot #1}
\newcommand{\curl}[1]{{\bf \nabla} \times #1}
\newcommand{\gradient}[1]{{\bf \nabla} #1}
\newcommand{\vect}[1]{{\bf #1}}
\newcommand{\am}{\bm{l}}


\section{Model Equations}
The evolution equations are
\begin{equation}
\ddt{\rho}+\divergence{\rho\vect{v}}=0,
\label{rho}
\end{equation}
\begin{equation}
\ddt{\vect{s}}+\divergence{\vect{v}\vect{s} + \gradient{p}}=\rho \vect{g},
\label{rhou}
\end{equation}
\begin{equation}
\ddt{\am}+\divergence{\vect{v}\am - \curl{\vect{x} p}}=\vect{x}\times\rho \vect{g},
\label{am}
\end{equation}
\begin{equation}
\ddt{\left(E+\tfrac{1}{2}\rho\phi\right)}+\divergence{\vect{v}\left(E+p+\rho\phi\right)}=\tfrac{1}{2}\left(\phi\ddt{\rho}-\rho\ddt{\phi}\right),
\label{E}
\end{equation}
and
\begin{equation}
\ddt{\tau}+\divergence{\tau\vect{u}}=0,
\label{tau}
\end{equation}
where $t$ is the time coordinate, $\rho$ is the mass density, $\vect{u}$ is the velocity, $\vect{s}$ is the inertial frame momentum density, $p$ is the gas pressure, 
$\vect{g}$ is the gravitational acceleration, 
$\am$ is the angular momentum relative to the coordinate center, 
$\vect{x}$ is the spatial coordinate relative to the origin, 
$E$ is the gas energy, $\phi$ is the gravitational potential, 
and $\tau$ is the entropy tracer.

The gas pressure, $p$, is taken as the ideal gas pressure for a monatomic gas, 
\begin{equation}
p = \left( \gamma - 1 ) \right) e, 
\end{equation}
where $\gamma = \tfrac{5}{3}$. The internal gas energy density, $e$, is computed using the dual energy formalism of \cite{BRYAN1995},
\begin{equation}
    e = 
\begin{cases}
    E - \tfrac{1}{2} \rho u^2, & \text{if } E - \tfrac{1}{2} \rho u^2 \geq \epsilon_1 E\\
    \tau^{\gamma},             & \text{else}
\end{cases},
\end{equation}
where $\epsilon_1 = 0.001$.

The angular momentum is defined as
\begin{equation}
\am = \vect{x} \times \rho \vect{u}.
\end{equation}


\section{Method}
We begin by reconstructing cell averaged values on the surface of the cells. Rather than reconstruct the vector of conserved quanties, 
 \begin{align}
 U &= \begin{bmatrix}
 	\rho \\
        \vect{s} \\
        \am \\
        E \\
        \tau
      \end{bmatrix},
\end{align}
we reconstruct 
 \begin{align}
 V &= \begin{bmatrix}
 	\rho \\
        \vect{u}\\
        \vect{z} \\
        E - \tfrac{1}{2} \rho u^2\\
        \tau
      \end{bmatrix},
\end{align}
where $\vect{z} := \frac{\am}{\rho} - \vect{x_c} \times \vect{s}$ is the spin angular momentum relative to a particular grid cell with center $\vect{x_c}$,
and then transform back to $U$ after reconstruction .

We use the piecewise parabolic method (PPM) of \cite{COLELLA1984}  to reconstruct values at each face, edge, and vertex. Here, we denote particlar faces, edges, and vertices using a $+$, $0$, or $-$ superscript
for each dimension. For example, $a^{+00}_{j k l} := a_{j k l}\left(x_{j+\tfrac{1}{2} k l}\right)$ is the value of $a$ at the right x-face the $ijk^\mathrm{th}$ cell. The five cell stencil required for PPM is
taken along the direction of each face, edge, or vertex. For example, computing $a^{+0-}_{j k l}$ uses the cell averaged values $a_{j + 2 k l - 2}$,$a_{j + 1 k l - 1}$,$a_{j k l}$,$u_{j - 1 k l + 1}$, and
$a_{j - 2 k l + 2}$.  

In this way, the PPM method is implemented as described in \cite{COLELLA1984}, including contact discontinuity detection, with one difference: before imposing monotonicity constraints
on the velocity field, we modify the reconstructed values to match as closely as possible the evolved angular momemtum field. This method resembles the one described by \cite{DESPRES2015}.

First we measure the amount of angular momentum, relative to cell centers, represented by the reconstructed linear momenta.
Summing over all $\pm$ combinations using Simpson's rule, the measured angular momentum for a cell is
\begin{equation}
\begin{split}
\vect{z}_{M, j k l} := 
& \frac{1}{216} \Sigma^{\pm\pm\pm}  \left(\vect{x}_{j\pm\tfrac{1}{2} k \pm\tfrac{1}{2} l\pm\tfrac{1}{2}}  - \vect{x}_{j k l}\right)\times \vect{\rho}^{\pm\pm\pm}_{j k l} \vect{u}^{\pm\pm\pm}_{j k l} + \\
& \frac{1}{54}  \ \Sigma^{\pm\pm} \left(\vect{x}_{j\pm\tfrac{1}{2}  k \pm\tfrac{1}{2} l}                  - \vect{x}_{j k l}\right)\times \vect{\rho}^{\pm\pm 0}_{j k l}  \vect{u}^{\pm\pm 0}_{j k l}  + \\ 
& \frac{1}{54}  \ \Sigma^{\pm\pm} \left(\vect{x}_{j\pm\tfrac{1}{2}  k                 l\pm\tfrac{1}{2}}   - \vect{x}_{j k l}\right)\times \vect{\rho}^{\pm 0\pm}_{j k l}  \vect{u}^{\pm 0\pm}_{j k l}  + \\
& \frac{1}{54}  \ \Sigma^{\pm\pm} \left(\vect{x}_{j                 k\pm\tfrac{1}{2}  l\pm\tfrac{1}{2}}   - \vect{x}_{j k l}\right)\times \vect{\rho}^{0\pm\pm }_{j k l}  \vect{u}^{0\pm\pm }_{j k l}  + \\
& \frac{2}{27}  \ \Sigma^{\pm}   \left(\vect{x}_{j\pm\tfrac{1}{2} k l}                                    - \vect{x}_{j k l}\right)\times \vect{\rho}^{\pm 0 0}_{j k l}   \vect{u}^{\pm 0 0}_{j k l}   +  \\
& \frac{2}{27}  \ \Sigma^{\pm}   \left(\vect{x}_{j k\pm\tfrac{1}{2} l}                                    - \vect{x}_{j k l}\right)\times \vect{\rho}^{0 \pm 0}_{j k l}   \vect{u}^{0 \pm 0}_{j k l}   +  \\
& \frac{2}{27}  \ \Sigma^{\pm}   \left(\vect{x}_{j k l\pm\tfrac{1}{2}}                                    - \vect{x}_{j k l}\right)\times \vect{\rho}^{0 0 \pm}_{j k l}   \vect{u}^{0 0 \pm}_{j k l}   
\end{split}
\end{equation}

Taking a Taylor expansion of the angular momentum relative to a cell center and integrating over the whole cell, we find
\begin{equation}
{\vect{z}}_{j k l} = \tfrac{1}{12} \epsilon_{n m q} \left(\frac{\partial s_{m}}{\partial x_n}\right)_{j k l} \Delta^2 \vect{e}_q + \vect{\mathcal{O}}(\Delta^3),
\label{beg_cor}
\end{equation}
where $\Delta$ is the even grid spacing and $\vect{e}_q$ is the $q^\mathrm{th}$ unit vector. Using this fact, we can correct the linear momenta to reproduce the
correct angular momenta by modifying the reconstruction according to
\begin{equation}
\vect{s}^{* p}_{j k l} = \vect{s}^{p}_{j k l} + \frac{6}{\Delta^2} \left(\vect{z}_{j k l} - \vect{z}_{M, j k l}\right) \times \left( \vect{x}^p - \vect{x}_{j k l} \right),
\end{equation}
where the $p$ superscripts refer to a give cell face, edge, or vertex.

Equation \ref{beg_cor} will reproduce a momentum field that matches the evolved angular momentum, however the reconstructed quantities may introduce new extrema to the solution. Denoting values on the
plus and minus sides of a cell in a given direction as $a^+$ and $a^-$, respectively, and the cell average as $\tilde{a}$, we can express the reconstructed parabola as 
\begin{equation}
\label{parabola}
a\left(x\right) := \tilde{a} + a_1 x + \left(a_2 - \tfrac{1}{12}\right) x^2,
\end{equation}
where 
\begin{equation}
a_1 = a^+ - a^-
\end{equation}
and
\begin{equation}
a_2 = 6 \left( \tfrac{1}{2}\left(a^+ + a^-\right) - \tilde{a} \right).
\end{equation}
Then, using the superbee limiter,
\begin{equation}
a_{1,\mathrm{lim},i} :=  \tfrac{1}{2}\left(\mathrm{sgn}\left(a_{i+1} - a_i \right) +  \mathrm{sgn}\left(a_i - a_{i-1} \right)\right) \mathrm{max}\left(\mathrm{min}\left(2 |a_{i+1} - a_i |, | a_i - a_{i-1}| \right),\mathrm{min}\left(|a_{i+1} - a_i |, 2| a_i - a_{i-1}| \right) \right)
\end{equation}
we constrain $a_1$ to the TVD limit, 
\begin{equation}
a_1 \rightarrow \mathrm{minmod}\left(a_1, a_{1,\mathrm{lim}} \right)
\end{equation}
The resulting left and right hand values may still not be TVD, 
due to the third term on the RHS of Equation (\ref{parabola}). Limiting $a_2$ to ensure $a^+_i$ is within the range $\left[a_i,a_{i+1}\right]$ and $a^-_i$ 
is within the range $\left[a_{i-1},a_i\right]$, we limit the magnitude of $a_2$ to $|a_{1,\mathrm{lim}} - a_1|$,
\begin{equation}
a_2 \rightarrow \mathrm{sgn}\left(a_2\right)\mathrm{min} \left( |a_2|, |a_{1,\mathrm{lim}} - a_1|\right).
\end{equation}
Using Equation (\ref{parabola}) we then obtain new values for $a^+$ and $a^-$, 
\begin{equation}
a^\pm \rightarrow \tilde{a} \pm \tfrac{1}{2} a_1 + \tfrac{1}{6} a_2.
\end{equation}


Then we apply the monotonicity constraints of \cite{COLELLA1984} to obtain the final values for $a^\pm$.



\bibliographystyle{plain}
\bibliography{references}


\end{document}
