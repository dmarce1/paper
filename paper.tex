\documentclass{aastex63}
\usepackage{amsmath}
\usepackage{bm}      % for \bm macro
\usepackage{natbib}
\setcitestyle{numbers}




\begin{document}


\newcommand{\ddt}[1]{\frac{\partial}{\partial t} #1} 
\newcommand{\divergence}[1]{{\bf \nabla} \cdot #1}
\newcommand{\curl}[1]{{\bf \nabla} \times #1}
\newcommand{\gradient}[1]{{\bf \nabla} #1}
\newcommand{\vect}[1]{{\bf #1}}
\newcommand{\am}{\bm{l}}


\section{Model Equations}
The evolution equations are
\begin{equation}
\ddt{\rho}+\divergence{\rho\vect{u}}=0,
\label{rho}
\end{equation}
\begin{equation}
\ddt{\vect{s}}+\divergence{\vect{u}\vect{s} + \gradient{p}}=\rho \vect{g} + \vect{\Omega} \times \vect{s},
\label{rhou}
\end{equation}
\begin{equation}
\ddt{\am}+\divergence{\vect{u}\am - \curl{\vect{x} p}}=\vect{x}\times\rho \vect{g} + \vect{\Omega} \times \am,
\label{am}
\end{equation}
\begin{equation}
\ddt{\left(E+\tfrac{1}{2}\rho\phi\right)}+\divergence{\vect{u}\left(E+p+\rho\phi\right)}=\tfrac{1}{2}\left(\phi\ddt{\rho}-\rho\ddt{\phi}\right),
\label{E}
\end{equation}
and
\begin{equation}
\ddt{\tau}+\divergence{\tau\vect{u}}=0,
\label{tau}
\end{equation}
where $t$ is the time coordinate, $\rho$ is the mass density, $\vect{u}$ is the velocity relative to the grid, $\vect{s}$ is the inertial frame momentum density, $p$ is the gas pressure, 
$\vect{g}$ is the gravitational acceleration, 
$\vect{\Omega}$ is the angular frequency of of the grid,
$\am$ is the angular momentum relative to the coordinate center, 
$\vect{x}$ is the spatial coordinate relative to the origin, 
$E$ is the gas energy, $\phi$ is the gravitational potential, 
and $\tau$ is the entropy tracer.

The gas pressure, $p$, is taken as the ideal gas pressure for a monatomic gas, 
\begin{equation}
p = \left( \gamma - 1 ) \right) e, 
\end{equation}
where $\gamma = \tfrac{5}{3}$. The internal gas energy density, $e$, is computed using the dual energy formalism of \cite{BRYAN1995},
\begin{equation}
    e = 
\begin{cases}
    E - \tfrac{1}{2} \rho u^2, & \text{if } E - \tfrac{1}{2} \rho u^2 \geq \epsilon_1 E\\
    \tau^{\gamma},             & \text{else}
\end{cases},
\end{equation}
where $\epsilon_1 = 0.001$.

The angular momentum is defined as
\begin{equation}
\am = \vect{x} \times \rho \vect{u}.
\end{equation}


\section{Method}
We begin by reconstructing cell averaged values on the surface of the cells. Rather than reconstruct the vector of conserved quanties, 
 \begin{align}
 U &= \begin{bmatrix}
 	\rho \\
        \vect{s} \\
        \am \\
        E \\
        \tau
      \end{bmatrix},
\end{align}
we reconstruct 
 \begin{align}
 V &= \begin{bmatrix}
 	\rho \\
        \frac{\vect{s}}{\rho} \\
        \frac{\am}{\rho} - \vect{x} \times \vect{s} \\
        E - \tfrac{1}{2} \frac{s^2}{\rho}\\
        \tau
      \end{bmatrix},
\end{align}
and then transform back to $U$ after reconstruction.

Using the piece-wise parabolic method (PPM) of \cite{COLELLA1984}, we reconstruct values at eight cell vertices, 
\begin{equation}
\begin{split}
V_{j \pm \tfrac{1}{2} k \pm \tfrac{1}{2} l \pm \tfrac{1}{2}} = & 
\frac{1}{2}\left(V_{ j k l } + V_{j \pm 1 k \pm 1 l \pm 1}\right) - \\
& \frac{1}{6}\left(
\mathrm{minmod}_2 \left(V_{j \pm 2 k \pm 2 l \pm 2} - V_{j \pm 1 k \pm 1 l \pm  1}, V_{j \pm 1 k \pm 1 l \pm 1} - V_{j k l }   \right) -
\mathrm{minmod}_2 \left(V_{j \pm 1 k \pm 1 l \pm 1} - V_{j k l}, V_{j k l } - V_{j \mp 1 k \mp 1 l \mp 1 }   \right)
\right),
\end{split}
\end{equation}
twelve cell edges, 
\begin{equation}
\begin{split}
V_{j \pm \tfrac{1}{2} k \pm \tfrac{1}{2} l } = & 
\frac{1}{2}\left(V_{ j k l } + V_{j \pm 1 k \pm 1 l}\right) - \\
& \frac{1}{6}\left(
\mathrm{minmod}_2 \left(V_{j \pm 2 k \pm 2 l} - V_{j \pm 1 k \pm 1 l \pm  1}, V_{j \pm 1 k \pm 1 l} - V_{j k l }   \right) -
\mathrm{minmod}_2 \left(V_{j \pm 1 k \pm 1 l} - V_{j k l}, V_{j k l } - V_{j \mp 1 k \mp 1 l}   \right)
\right),
\end{split}
\end{equation}
\begin{equation}
\begin{split}
V_{j \pm \tfrac{1}{2} k l \pm \tfrac{1}{2}} = & 
\frac{1}{2}\left(V_{ j k l } + V_{j \pm 1 k l \pm 1}\right) - \\
& \frac{1}{6}\left(
\mathrm{minmod}_2 \left(V_{j \pm 2 k l \pm 2} - V_{j \pm 1 k l \pm  1}, V_{j \pm 1 k l \pm 1} - V_{j k l }   \right) -
\mathrm{minmod}_2 \left(V_{j \pm 1 k l \pm 1} - V_{j k l}, V_{j k l } - V_{j \mp 1 k l \mp 1 }   \right)
\right),
\end{split}
\end{equation}
\begin{equation}
\begin{split}
V_{j k \pm \tfrac{1}{2} l \pm \tfrac{1}{2}} = & 
\frac{1}{2}\left(V_{ j k l } + V_{j k \pm 1 l \pm 1}\right) - \\
& \frac{1}{6}\left(
\mathrm{minmod}_2 \left(V_{j k \pm 2 l \pm 2} - V_{j k \pm 1 l \pm  1}, V_{j k \pm 1 l \pm 1} - V_{j k l }   \right) -
\mathrm{minmod}_2 \left(V_{j k \pm 1 l \pm 1} - V_{j k l}, V_{j k l } - V_{j k \mp 1 l \mp 1 }   \right)
\right),
\end{split}
\end{equation}
and six cell faces,
\begin{equation}
\begin{split}
V_{j \pm \tfrac{1}{2} k l } = & 
\frac{1}{2}\left(V_{ j k l } + V_{j \pm 1 k l}\right) - \\
& \frac{1}{6}\left(
\mathrm{minmod}_2 \left(V_{j \pm 2 k l} - V_{j \pm 1 k l }, V_{j \pm 1 k l} - V_{j k l }   \right) -
\mathrm{minmod}_2 \left(V_{j \pm 1 k l} - V_{j k l}, V_{j k l } - V_{j \mp 1 k l}   \right)
\right),
\end{split}
\end{equation}
\begin{equation}
\begin{split}
V_{j  k \pm \tfrac{1}{2} l} = & 
\frac{1}{2}\left(V_{ j k l } + V_{j k \pm 1 l}\right) - \\
& \frac{1}{6}\left(
\mathrm{minmod}_2 \left(V_{j k \pm 2 l} - V_{j k \pm 1 l }, V_{j k \pm 1 l } - V_{j k l }   \right) -
\mathrm{minmod}_2 \left(V_{j k \pm 1 l} - V_{j k l}, V_{j k l } - V_{j k \mp 1 l }   \right)
\right),
\end{split}
\end{equation}
\begin{equation}
\begin{split}
V_{j  k l \pm \tfrac{1}{2}} = & 
\frac{1}{2}\left(V_{ j k l } + V_{j k l \pm 1}\right) - \\
& \frac{1}{6}\left(
\mathrm{minmod}_2 \left(V_{j k l \pm 2} - V_{j k l \pm  1}, V_{j k l \pm 1} - V_{j k l }   \right) -
\mathrm{minmod}_2 \left(V_{j k l \pm 1} - V_{j k l}, V_{j k l } - V_{j k l \mp 1 }   \right)
\right),
\end{split}
\end{equation}
where
\begin{equation}
\mathrm{minmod}_2 \left( a, b \right) := \mathrm{minmod} \left( 2 \ \mathrm{minmod}\left(a,b\right), \tfrac{1}{2}\left(a + b\right)\right).
\end{equation}
Here, $\mathrm{minmod}$ is the ubiquitous minmod operator, 
\begin{equation}
\mathrm{minmod}\left(a,b\right) = \tfrac{1}{2}\left(\mathrm{sgn}\left(a\right)+\mathrm{sgn}\left(b\right)\right) \mathrm{min}\left( | a |, | b | \right).
\end{equation}

Now we split the shared reconstructed values into
\begin{equation}
V^{\pm \pm \pm}_{j k l} \rightarrow V_{j\pm\tfrac{1}{2} k\pm\tfrac{1}{2} l\pm\tfrac{1}{2}},
\end{equation}
\begin{equation}
V^{\pm \pm 0}_{j k l} \rightarrow V_{j\pm\tfrac{1}{2} k\pm\tfrac{1}{2} l},
\end{equation}
\begin{equation}
V^{\pm 0 \pm}_{j k l} \rightarrow V_{j\pm\tfrac{1}{2} k l\pm\tfrac{1}{2}},
\end{equation}
\begin{equation}
V^{0 \pm \pm}_{j k l} \rightarrow V_{j k\pm\tfrac{1}{2} l\pm\tfrac{1}{2}},
\end{equation}
\begin{equation}
V^{\pm 0 0}_{j k l} \rightarrow V_{j\pm\tfrac{1}{2} k l},
\end{equation}
\begin{equation}
V^{0 \pm 0}_{j k l} \rightarrow V_{j k\pm\tfrac{1}{2} l},
\end{equation}
and
\begin{equation}
V^{0 0 \pm}_{j k l} \rightarrow V_{j k l\pm\tfrac{1}{2}}.
\end{equation}


The PPM limiter is defined as
\begin{align}
  \begin{bmatrix}
 	a_R \\
        a_L
  \end{bmatrix} \rightarrow
  \begin{bmatrix}
 	\begin{cases}
             a_0 &  \mathrm{if} \ \left(a_R - a_0\right)\left(a_0 - a_L\right) < 0 \\
             3 a_0 - 2 a_L & \mathrm{if}  \ -\frac{\left(a_R - a_L\right)}{6} < \left(a_0 - \tfrac{1}{2}\left(a_R + a_L\right)\right) \\
             a_R &  \mathrm{else}
        \end{cases} \\
\\
         \begin{cases}
             a_0 &  \mathrm{if} \ \left(a_R - a_0\right)\left(a_0 - a_L\right) < 0 \\
             3 a_0 - 2 a_R & \mathrm{if} \ \frac{\left(a_R - a_L\right)}{6} < \left(a_0 - \tfrac{1}{2}\left(a_R + a_L\right)\right) \\
             a_L &  \mathrm{else}
        \end{cases}
  \end{bmatrix}
\end{align}



\bibliographystyle{plain}
\bibliography{references}


\end{document}
